% Preamble
\documentclass[10pt]{article}
\title{Supplementary Material}
\author{Yihang Cheng}
\date{\today}
% Packages
\usepackage{amsmath}
\usepackage{graphicx}
\usepackage{multirow}
\usepackage{array}
\usepackage[colorlinks=true, linkcolor=blue, urlcolor=blue, citecolor=blue]{hyperref}
\usepackage{listings}
\usepackage{xcolor}
\usepackage{threeparttable}
% Document
\begin{document}
\maketitle
\newpage
\tableofcontents
\newpage
\section{Software tools developed in this study}\label{sec:software-develope}
The analysis algorithms and methodologies employed in this paper were integrated into two tool sets:

\begin{itemize}
    \item  \href{https://github.com/hcyvan/epiLungCancer/tree/main/methytools}{\textit{methytools}}
    \item \href{https://github.com/hcyvan/pattools}{\textit{pattools}}
\end{itemize}

\subsection{methytools}\label{sec:software-develope-methytools}

methytools is a toolkit developed for processing BS-seq methylation data.

\subsubsection{mcomppost percentile}

\emph{mcomppost percentile} can be used to calculate the percentile of each CpGs and add percentile columns to the DMCs files.

\begin{lstlisting}[language=sh, caption={}]
mcomppost percentile -i ./dmc.Rest.vs.CTL.txt \
                     -m ./merge.d3.all.bed.gz \
                     -t ./sample.CTL.txt \
                     -o ./dmc.Rest.vs.CTL.percentile.txt
\end{lstlisting}

\subsection{pattools}\label{sec:software-develope-pattools}

pattools is a toolkit based on PAT format\cite{Loyfer2024.05.08.593132} data, inheriting both existing
methylation algorithms and incorporating new ones.
    {pattools} is currently under development, and the algorithms from this toolkit are primarily utilized
in this study. The software will be introduced in a forthcoming paper.

\subsubsection{pattools vector}

\section{Selection of subtype-specific DMCs}\label{sec:selection-of-dmcs}

\subsection{One-vs-Rest}

We employed the One-vs-Rest strategy to identify specific differentially methylated CpGs (DMCs) across various
subtypes of lung cancer, using MOABS\cite{sun_moabs_2014} for preliminary screening. The target group and Background
group are shown in the following table\ref*{tab:grouping}.

\begin{table}[htbp]
    \begin{center}
        \caption{ Grouping of One-vs-Rest }
        \begin{tabular}{|l|l|}
            \hline
            Target & Background            \\
            \hline
            CTL    & LUAD, LUSC, LCC, SCLC \\
            LUAD   & CTL, LUSC, LCC, SCLC  \\
            LUSC   & CTL, LUAD, LCC, SCLC  \\
            LCC    & CTL, LUAD, LUSC, SCLC \\
            SCLC   & CTL, LUAD, LUSC, LCC  \\
            \hline
        \end{tabular}
    \end{center}
    \label{tab:grouping}

\end{table}

\subsection{MOABS mcomp}

Using the default parameters of MOABS, we obtained a large number of DMCs in each comparison group (Table \ref{tab:table_dmc_ount}).
Subsequently, duplicated differentially methylated CpG sites (DMCs) across multiple groups were excluded, leaving
only those specific to CTL, LUAD, LUSC, LCC, and SCLC.

\begin{table}[htbp]
    \begin{center}
        \caption{The count of DMCs find by MOABS}
        \begin{threeparttable}
            \begin{tabular}{|l|p{0.2\textwidth}|p{0.2\textwidth}|}
                \hline
                \multirow{2}{*}{Group} & \multicolumn{2}{c|}{Filtered DMCs\tnote{1}}           \\
                \cline{2-3}            & Befor                                       & After   \\
                \hline
                CTL vs. Rest           & 584239                                      & 139129  \\
                LUAD vs. Rest          & 95999                                       & 30220   \\
                LUSC vs. Rest          & 427525                                      & 318345  \\
                LCC vs. Rest           & 1361568                                     & 1068507 \\
                SCLC vs. Rest          & 3022977                                     & 2173620 \\
                \hline
            \end{tabular}

            \begin{tablenotes}
                \item[1] Exclude the following DMCs:
                1. Those that appear in more than one group.
                2. Those located on the sex chromosomes
            \end{tablenotes}

        \end{threeparttable}
    \end{center}
    \label{tab:table_dmc_ount}
\end{table}

\subsection{\textit{p}-th percentile algorithm}

A more stringent strategy was then used for further screening of DMCs. We evaluate the
\textit{p}-th percentile of low-methylated group and the (\textit{100 - p})-th percentile
of high-methylated group. In hypermethylated DMCs, the \textit{p}-th percentile of the target
group and the (\textit{100 - p})-th percentile of the background are calculated; conversely,
in hypomethylated DMCs, this relationship is reversed. If the \textit{p}-th percentile of the
low-methylated group is less than or equal to (\textit{100-p})-th percentile of the high-methylated
group, we assert that this DMCs adheres to our screening criteria (Figure \ref*{fig:dmcPC}).

\begin{figure}[htbp]
    \centering
    \includegraphics[width=0.8\linewidth]{./img/dmcPercentileCut.png}
    \caption{The 85-th percentile of target and background samples in a hypomethylated DMCs}
    \label{fig:dmcPC}
\end{figure}

Utilizing \emph{mcomppost}, we conducted an analysis of the DMCs within each group. As
the \textit{p}-th percentile increased, the number of retained DMCs decreased significantly
(Figure \ref*{fig:dmcPK}). By selecting an appropriate percentile (\textit{p}) threshold,
we can identify DMCs that more stringently adhere to
group specificity.


\begin{figure}[htbp]
    \centering
    \includegraphics[width=1\linewidth]{./img/dmcPercentileVsKeepProp}
    \caption{The remaining DMC propertion after filtering with different percentiles}
    \label{fig:dmcPK}
\end{figure}

\section{Deconvolution}\label{sec:deconvolution}
Three deconvolution algorithms (lofer\cite{Loyfer2024.05.08.593132}, moss \cite{moss_comprehensive_2018} 
and sun\cite{sun_plasma_2015}) have been utilized to perform the deconvolution of the tissue samples.

To mitigate the interference caused by the presence of numerous cell/tissue types in
lung tissue deconvolution, we manually selected specific cell/tissue types that are potentially
implicated in lung cancer for this analysis (Table \ref*{tab:deconvolution}). For clarification
on the cell/tissue types, \textit{pattools deconv-helper} can be used to view the corresponding
cell/tissue types for each category.

\begin{table}[htbp]
    \begin{center}
        \caption{Cell/Tissue types used by each algorithm }
        \begin{tabular}{|>{\raggedright}m{2cm}|>{\raggedright}m{4cm}|>{\raggedright}m{4cm}|c|}
            \hline
            \textbf{Method}                         & \textbf{Refined group}                      & \textbf{Group}           & \textbf{Note} \\
            \hline
            \multirow{6}{*}{Sun \textit{et al.}}    & Lung cells                                  & Lungs                    &               \\
            \cline{2-4}                             & Neural and endocrine cells                  & Brain                    &               \\
            \cline{2-4}                             & \multirow{3}{*}{Immune cells}               & T-cells                  &               \\
            \cline{3-4}                             &                                             & B-cells                  &               \\
            \cline{3-4}                             &                                             & Neutrophils              &               \\
            \cline{2-4}                             & Others                                      & Heart                    &               \\
            \hline
            \multirow{6}{*}{Moss \textit{et al.}}   & Lung cells                                  & LungCells                &               \\
            \cline{2-4}                             & \multirow{2}{*}{Neural and endocrine cells} & CorticalNeurons          &               \\
            \cline{3-4}                             &                                             & PancreaticBetaCells      &               \\
            \cline{2-4}                             & \multirow{6}{*}{Immune cells}               & Cd4tCells                &               \\
            \cline{3-4}                             &                                             & Cd8tCells                &               \\
            \cline{3-4}                             &                                             & BCells                   &               \\
            \cline{3-4}                             &                                             & NkCells                  &               \\
            \cline{3-4}                             &                                             & Monocytes                &               \\
            \cline{3-4}                             &                                             & Neutrophils              &               \\
            \cline{2-4}                             & \multirow{2}{*}{Others}                     & VascularEndothelialCells &               \\
            \cline{3-4}                             &                                             & LeftAtrium               &               \\
            \hline
            \multirow{6}{*}{Loyfer \textit{et al.}} & \multirow{2}{*}{Lung cells}                 & Lung-Ep-Alveo            &               \\
            \cline{3-4}                             &                                             & Lung-Ep-Bron             &               \\
            \cline{2-4}                             & \multirow{5}{*}{Neural and endocrine cells} & Neuron                   &               \\
            \cline{3-4}                             &                                             & Oligodend                &               \\
            \cline{3-4}                             &                                             & Pancreas-Alpha           &               \\
            \cline{3-4}                             &                                             & Pancreas-Beta            &               \\
            \cline{3-4}                             &                                             & Pancreas-Delta           &               \\
            \cline{2-4}                             & \multirow{5}{*}{Immune cells}               & Blood-T                  &               \\
            \cline{3-4}                             &                                             & Blood-B                  &               \\
            \cline{3-4}                             &                                             & Blood-NK                 &               \\
            \cline{3-4}                             &                                             & Blood-Mono+Macro         &               \\
            \cline{3-4}                             &                                             & Blood-Granul             &               \\
            \cline{2-4}                             & \multirow{2}{*}{Others}                     & Endothel                 &               \\
            \cline{3-4}                             &                                             & Heart-Fibro              &               \\
            \cline{3-4}                             &                                             & Head-Neck-Ep             &               \\
            \hline
        \end{tabular}
    \end{center}
    \label{tab:deconvolution}
\end{table}

For BS-seq data, wgbstools\cite{Loyfer2024.05.08.593132} were initially employed to convert the 
aligned BAM format data into PAT format. Subsequently, the methylation metrics required by the corresponding 
algorithm were extracted from the PAT format for deconvolution analysis. The algorithm utilized in this 
study for deconvolution was NNLS. All three deconvolution algorithms discussed in this article were 
implemented using \textit{pattools deconv}.

This code performs the deconvolution of a sample using algorithm developed by Sun \textit{et al.} with \textit{pattools deconv}.

\begin{lstlisting}[language=sh, caption={}]
pattools deconv -m sun -g hg38 \
        -c /path/to/references/hg38/CpG.bed.gz \
        -p sample.pat.gz -o sample.sun.tsv \
        --include Lungs Heart Brain T-cells B-cells Neutrophils
\end{lstlisting}

\section{Abbreviations}\label{sec:abbr}
\begin{itemize}
    \item  \textbf{BS-seq}: Bisulfite sequencing
    \item  \textbf{NNLS}: non-negative least squares linear regression
\end{itemize}
\section{Code availability}\label{sec:code}

The analysis code of this artile a available in Github repository
\href{https://github.com/hcyvan/epiLungCancer}{epiLungCancer}. The sub project of epiLungCancer,
\href{https://github.com/hcyvan/epiLungCancer/tree/main/methytools}{\textit{methytools}},  was developed
to process BS-seq methylation data. Additionally, some of the analyses in this paper utilize code
from \href{https://github.com/hcyvan/pattools}{pattools}, which will be discussed in detail in
separate publications.



\bibliographystyle{plain}
\bibliography{SupplymentaryMaterialReference}
\end{document}