% Preamble
\documentclass[10pt]{article}
\title{Supplementary Material}
\author{Yihang Cheng}
\date{\today}
% Packages
\usepackage{amsmath}
\usepackage{graphicx}
\usepackage{multirow}
\usepackage{array}
\usepackage[colorlinks=true, linkcolor=blue, urlcolor=blue, citecolor=blue]{hyperref}
\usepackage{listings}
\usepackage{xcolor}
% Document
\begin{document}
\maketitle


\section{Method}\label{sec:method}

\subsection{Find the differentially methylated CpGs}\label{subsec:find-the-differentially-methylated-cpgs-(dmcs)}

We employed the One-vs-Rest strategy to identify specific differentially methylated CpGs (DMCs) across various
subtypes of lung cancer, using MOABS\cite{sun_moabs_2014} for preliminary screening. The target group and Background
group are shown in the following table\ref*{tab:grouping}.

\begin{table}[htbp]
    \begin{center}
        \caption{ Grouping of One-vs-Rest }
        \begin{tabular}{|c|c|}
            \hline
            Target & Background            \\
            \hline
            CTL    & LUAD, LUSC, LCC, SCLC \\
            LUAD   & CTL, LUSC, LCC, SCLC  \\
            LUSC   & CTL, LUAD, LCC, SCLC  \\
            LCC    & CTL, LUAD, LUSC, SCLC \\
            SCLC   & CTL, LUAD, LUSC, LCC  \\
            \hline
        \end{tabular}
    \end{center}
    \label{tab:grouping}
\end{table}

Using the default parameters of MOABS, we obtained a large number of DMCs in each comparison group (Table \ref{tab:table_dmc_ount}).
Subsequently, duplicated differentially methylated CpG sites (DMCs) across multiple groups were excluded, leaving
only those specific to CTL, LUAD, LUSC, LCC, and SCLC.

\begin{table}[htbp]
    \begin{center}
        \caption{The count of DMCs find by MOABS}
        \begin{tabular}{|c|c|c|}
            \hline
            \multirow{2}{*}{Group} & \multicolumn{2}{c|}{Remove multiple DMCs}           \\
            \cline{2-3}            & Befor                                     & After   \\
            \hline
            CTL vs. Rest           & 584239                                    & 145200  \\
            LUAD vs. Rest          & 95999                                     & 31299   \\
            LUSC vs. Rest          & 427525                                    & 326642  \\
            LCC vs. Rest           & 1361568                                   & 1090432 \\
            SCLC vs. Rest          & 3022977                                   & 2349828 \\
            \hline
        \end{tabular}
    \end{center}
    \label{tab:table_dmc_ount}
\end{table}

A more stringent strategy was then used for further screening of DMCs. We evaluate the \textit{p}-th percentile of low-methylated group
and the \textit{1 - p}-th percentile of high-methylated group. In hypermethylated DMCs, the \textit{p}-th percentile of the target
group and the \textit{1 - p}-th percentile of the background are calculated; conversely, in hypomethylated DMCs, this relationship is
reversed. If the \textit{p}-th percentile of the low-methylated group is not less than or equal to \textit{1-p}-th percentile of the
high-methylated group, we assert that this DMCs adheres to our screening criteria (Figure \ref*{fig:dmcPC}).

\begin{figure}[htbp]
    \centering
    \includegraphics[width=0.8\linewidth]{./img/dmcPercentileCut.png}
    \caption{The 85-th percentile of target and background samples in a hypomethylated DMCs}
    \label{fig:dmcPC}
\end{figure}

In this study, we developed MethyTools, a comprehensive toolbox for processing BS-seq data. \emph{mcomppost percentile} can be used to calculate
the percentile of each CpGs.

\begin{lstlisting}[language=sh, caption={}]
mcomppost percentile -i ./dmc.Rest.vs.CTL.txt \
                     -m ./merge.d3.all.bed.gz \
                     -t ./sample.CTL.txt \
                     -o ./dmc.Rest.vs.CTL.percentile.txt
\end{lstlisting}

Utilizing \emph{mcomppost}, we conducted an analysis of the DMCs within each group. As the percentile increased, the number of retained DMCs 
decreased significantly (Figure \ref*{fig:dmcPK}). By selecting an appropriate percentile (\textit{p}) threshold, we can identify DMCs that more stringently adhere to 
group specificity.


\begin{figure}[htbp]
    \centering
    \includegraphics[width=1\linewidth]{./img/dmcPercentileVsKeepProp}
    \caption{The remaining DMC propertion after filtering with different percentiles}
    \label{fig:dmcPK}
\end{figure}

\section{Code availability}\label{sec:code}

The analysis code of this artile a available in Github repository \href{https://github.com/hcyvan/epiLungCancer}{epiLungCancer}. The
sub project of epiLungCancer, \href{https://github.com/hcyvan/epiLungCancer/tree/main/methytools}{\textit{methytools}},  was developed to process
BS-seq methylation data.



\bibliographystyle{plain}
\bibliography{SupplymentaryMaterialReference}
\end{document}